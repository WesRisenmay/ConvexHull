\documentclass{article}
\usepackage{amsmath, amsthm, amssymb, bm}
\usepackage{graphicx}
\usepackage{makeidx}
\makeidx
\setlength\parindent{0pt} % Removes all indentation from paragraphs

\renewcommand{\labelenumi}{\alph{enumi}.} % Make numbering in the enumerate environment by letter rather than number (e.g. section 6)
\begin{document}
\input{./title_sheet.tex}
\tableofcontents{}
{
  \section{Objective}
  \section{ConvexHull Overview}
  \section{Algorithm Explaination}
  \subsection{Brute Force}
  \subsection{QuickHull}
  \section{Algorithm Implementation}
  \subsection{Brute Force}
  As the name implies the brute force solution to the convexhule problem is very straight forward, as shown below in Figure ~\ref{fig:bruteforce}.
  \begin{figure}
    \includegraphics[width=16cm,height=17cm]{bruteforce.png}
    \caption{Brute Force Convex Hull}
    \label{fig:bruteforce}
\end{figure}
  \subsection{QuickHull}
  \section{Analysis of Complexity}
  \subsection{Brute Force}
  \subsection{QuickHull}
  \subsection{Expected Outcome}
  \section{Automated Testing}
  \subsection{Algorithm Correctness}
  \subsection{Data Generation}
  \subsection{Timing Execution}
  \section{Results}
  \section{Conclusion}



  %----------------------------------------------------------------------------------------
  % BIBLIOGRAPHY
  %----------------------------------------------------------------------------------------

  \bibliographystyle{unsrt}

  \bibliography{sample}

  %----------------------------------------------------------------------------------------


\end{document}
