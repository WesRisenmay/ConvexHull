\documentclass{article}
\usepackage{amsmath, amsthm, amssymb, bm}
\usepackage{graphicx}
\usepackage{makeidx}
\makeidx
\setlength\parindent{0pt} % Removes all indentation from paragraphs

\renewcommand{\labelenumi}{\alph{enumi}.} % Make numbering in the enumerate environment by letter rather than number (e.g. section 6)
\begin{document}
%%%%%%%%%%%%%%%%%%%%%%%%%%%%%%%%%%%%%%%%%
% University Assignment Title Page 
% LaTeX Template
% Version 1.0 (27/12/12)
%
% This template has been downloaded from:
% http://www.LaTeXTemplates.com
%
% Original author:
% WikiBooks (http://en.wikibooks.org/wiki/LaTeX/Title_Creation)
%
% License:
% CC BY-NC-SA 3.0 (http://creativecommons.org/licenses/by-nc-sa/3.0/)
% 
% Instructions for using this template:
% This title page is capable of being compiled as is. This is not useful for 
% including it in another document. To do this, you have two options: 
%
% 1) Copy/paste everything between \begin{document} and \end{document} 
% starting at \begin{titlepage} and paste this into another LaTeX file where you 
% want your title page.
% OR
% 2) Remove everything outside the \begin{titlepage} and \end{titlepage} and 
% move this file to the same directory as the LaTeX file you wish to add it to. 
% Then add \input{./title_page_1.tex} to your LaTeX file where you want your
% title page.
%
%%%%%%%%%%%%%%%%%%%%%%%%%%%%%%%%%%%%%%%%%

%----------------------------------------------------------------------------------------
% PACKAGES AND OTHER DOCUMENT CONFIGURATIONS
%----------------------------------------------------------------------------------------
\begin{titlepage}

\newcommand{\HRule}{\rule{\linewidth}{0.5mm}} % Defines a new command for the horizontal lines, change thickness here
\vspace*{40mm}
\center % Center everything on the page
 
 %----------------------------------------------------------------------------------------
 %  HEADING SECTIONS
 %----------------------------------------------------------------------------------------

 \textsc{\LARGE Portland State University}\\[1.5cm] % Name of your university/college
 \textsc{\Large CS350 }\\[0.5cm] % Major heading such as course name
 \textsc{\large Algorithms And Complexity}\\[0.5cm] % Minor heading such as course title

 %----------------------------------------------------------------------------------------
 %  TITLE SECTION
 %----------------------------------------------------------------------------------------

 \HRule \\[0.4cm]
 { \huge \bfseries ConvexHull Analysis:\\ BruteForce vs. QuickHull}\\[0.4cm] % Title of your document
 \HRule \\[1.5cm]
  
  %----------------------------------------------------------------------------------------
  % AUTHOR SECTION
  %----------------------------------------------------------------------------------------

  \begin{minipage}{0.4\textwidth}
  \begin{flushleft} \large
  \emph{Author:}\\
  Wes \textsc{Risenmay}\\ % Your name
  Josh \textsc{Willhite} % Your name
  \end{flushleft}
  \end{minipage}
  ~
  \begin{minipage}{0.4\textwidth}
  \begin{flushright} \large
  \emph{Instructor:} \\
  Dr. Andrew \textsc{Black} % Supervisor's Name
  \end{flushright}
  \end{minipage}\\[4cm]

  % If you don't want a supervisor, uncomment the two lines below and remove the section above
  %\Large \emph{Author:}\\
  %John \textsc{Smith}\\[3cm] % Your name

  %----------------------------------------------------------------------------------------
  % DATE SECTION
  %----------------------------------------------------------------------------------------

  {\large \today}\\[3cm] % Date, change the \today to a set date if you want to be precise

  %----------------------------------------------------------------------------------------
  % LOGO SECTION
  %----------------------------------------------------------------------------------------

  %\includegraphics{Logo}\\[1cm] % Include a department/university logo - this will require the graphicx package
   
   %----------------------------------------------------------------------------------------

   \vfill % Fill the rest of the page with whitespace

   \end{titlepage}

\tableofcontents{}
{
  \section{Objective}
  \section{ConvexHull Overview}
  \section{Algorithm Explaination}
  \subsection{Brute Force}
  \subsection{QuickHull}
  \section{Algorithm Implementation}
  \subsection{Brute Force}
  As the name implies the brute force solution to the convexhule problem is very straight forward, as shown below in Figure ~\ref{fig:bruteforce}.
  \begin{figure}
    \includegraphics[width=16cm,height=17cm]{bruteforce.png}
    \caption{Brute Force Convex Hull}
    \label{fig:bruteforce}
\end{figure}
  \subsection{QuickHull}
  \section{Analysis of Complexity}
  \subsection{Brute Force}
  \subsection{QuickHull}
  \subsection{Expected Outcome}
  \section{Automated Testing}
  \subsection{Algorithm Correctness}
  \subsection{Data Generation}
  \subsection{Timing Execution}
  \section{Results}
  \section{Conclusion}



  %----------------------------------------------------------------------------------------
  % BIBLIOGRAPHY
  %----------------------------------------------------------------------------------------

  \bibliographystyle{unsrt}

  \bibliography{sample}

  %----------------------------------------------------------------------------------------


\end{document}
