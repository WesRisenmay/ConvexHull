\documentclass[a4paper , 12pt]{article}
\usepackage{amsmath, amsthm, amssymb, bm}
\usepackage{graphicx}
\usepackage[letterpaper, left=2cm,right=3cm,top=1.5cm,bottom=1.5cm]{geometry}
\title{CS350 Algorithms And Complexity Term Project Proposal}
\date{February 20, 2014}
\author{Wes Risenmay, Josh Willhite}

\begin{document}
\maketitle
{

\begin{enumerate}
\item
  \textbf{Topic: Convex Hull Problem (Brute Force vs. Quickhull Algorithms)}

Since Convex Hull problems are modeling a physical system it should be really interesting and educational to create visualizations of exactly how the two algorithms work. Also we've spent a lot of time in class learning about sorting algorithms and data structures were well covered earlier in our academic careers. 
   
\item
  \textbf{Language: Java}

Java as a language for implementing the algorithms seems natural since both of us have development environments and are familiar with it.  Also there are tons of libraries and resources for automated testing as well as visualization. 
\item
  \textbf{Features}:
  \begin{enumerate}
    \item Description of both algorithms, along with worked examples for very small $n$.
    \item Source code for implementation of algorithms.
    \item Description of automated testing. 
      \begin{enumerate}
        \item
          Check for correctness of algorithms against known implementations.
        \item
          Generation of input data.
        \item
          Time efficiency for varied values of $n$
      \end{enumerate}
    \item Analysis of complexity.
      \begin{enumerate}
        \item
          Exploration of cases where brute force may be faster than quick hull.
      \end{enumerate}
    \item Results.
      \begin{enumerate}
        \item
          Expected vs. actual results.
        \item
          Explanation of any discrepancies.
        \end{enumerate}
    \item Implementation challenges along with what we learned.
  \end{enumerate}
\item
  \textbf{Time Plan}
  \begin{center}
  \includegraphics[width=15cm,height=3cm]{schedule.png}
  \end{center}
\item
  \textbf{Collaboration Plan}
  \begin{enumerate}
    \item
      Weekly meetings Tuesdays after class to pair program. 
    \item
      Coordination of source code and report text via public github repository. 
    \item
      Work independently and teleconference meetings as necessary.  
  \end{enumerate}
\end{enumerate}
\end{document}
